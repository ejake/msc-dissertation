\chapter{Conclusions}
\label{chap:conclusions}
%\minitoc


\begin{itemize}
 \item Although the genetic operators attempt to explore by extending the search space to avoid local optimum, these operators by themselves do not improve the solutions. Hence, it is necessary to focus on local search in order to improve the quality solutions and the convergence.
 \item When the instances size increase, the distance between the quality of solutions found by the memetic algorithm and the solutions computed by others algorithms increase as well.
 \item Not only the memetic algorithm results are better than other algorithms' results, but also its outcomes have also less variability.
 \item The memetic algorithm consumes similar running time in large instances than the basic genetic algorithm, but its results are better.
 \item Although the hybridization approach increases the computational time cost, this effort is rewarded in many cases given that the quality of the solutions is better than the oune obtained when applying the rollout algorithm alone. 
 \item The rollout algorithm is inexpensive in comparison with other techniques.
 \item The genetic algorithm is more inexpensive than the memetic one in small and medium instances.
 \item The rollout is better than the basic genetic algorithm since it produces better results by employing less computational resources.
%resources 
 \item Local search is the most time-consuming component.
 \item The expected distance evaluation consumes a lot of execution time. An efficient approximation improves the algorithm performance.
%convergence
 \item In the behaviour of the genetic algorithms with small and medium instances, we observe that if local search is applied, the evolutionary algorithm stops when the number of iterations is achieved, rather than with the basic genetic algorithm which often completes the maximum counting of iterations without a significant change. This contrasts with the behaviour that we expected, since local search accelerates the solution convergence in the memetic algorithm. However, we do observe this behavior in large instances.
 %Contrast
 \item Despite that the basic genetic algorithm often expends less time, the hybrid approach finds a solution employing less time for many large instances. The memetic algorithm can perform less iterations since it obtains an unbeatable solution earlier. This induces the algorithm to stop, since it accomplishes the number of iterations whithout a significant change.
 
\end{itemize}




%\section*{Perspectives}

% To evaluate expected distance consume a lot of computational time, so an efficient approximation improves the algorithm performance.

% Implement algorithm to work in parallel computing
